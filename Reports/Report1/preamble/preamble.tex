\usepackage{setspace}

\usepackage{fancyhdr}
\setlength{\parindent}{0in}

% Page Formatting
\pagenumbering{arabic} %\pagenumbering{gobble}
\onehalfspacing %doublespacing
\pagestyle{fancy}
%\usepackage{pdfpages}

% Heading Formatting
\headheight 32pt

% Link Formatting
\usepackage{hyperref}
\hypersetup{
	colorlinks,
	allcolors=black
	%citecolor=black,
	%filecolor=black,
	%linkcolor=black,
	%urlcolor=black
}

\usepackage{mdframed}

% Code Formatting
\usepackage{listings}
\lstset
{ %Formatting for code
	basicstyle=\footnotesize\ttfamily,
	numbers=left,
	%caption={},
	%title={},
	stepnumber=1,
	showstringspaces=false,
	tabsize=1,
	breaklines=true,
	breakatwhitespace=false,
	frame=lines,
	xleftmargin=2em,
	framexleftmargin=1.5em,
	commentstyle=\color{commentsColor}\ttfamily,
  	stringstyle=\color{stringColor}\ttfamily,
	keywordstyle=\color{keywordsColor}\bfseries,
}
\newcommand{\lstprompt}{>>>}
\newcommand{\numberwithprompt}[1]{\footnotesize\ttfamily\selectfont \lstprompt}


% https://tex.stackexchange.com/questions/340232/how-insert-a-character-at-the-begin-of-every-line-from-a-source-code
\lstdefinestyle{console} {
	basicstyle=\footnotesize\ttfamily,
	numbers=left,
	stepnumber=1,
	showstringspaces=false,
	tabsize=1,
	breaklines=true,
	breakatwhitespace=false,
	frame=single,
	xleftmargin=2em,
	framexleftmargin=2.5em,
	%backgroundcolor=\color{gray!55},
	numberstyle=\numberwithprompt,
	}

\lstdefinestyle{psuedo}{ %this is the stype
	mathescape=true,
	frame=tB,
	numbers=left, 
	numberstyle=\tiny,
	basicstyle=\footnotesize\ttfamily, 
	keywordstyle=\color{black}\bfseries\em,
	keywords={,input, output, return, datatype, function, in, if, else, foreach, while, begin, end, } %add the keywords you want, or load a language as Rubens explains in his comment above.
	numbers=left,
	xleftmargin=.04\textwidth,
	#1 % this is to add specific settings to an usage of this environment (for instnce, the caption and referable label)
}

% \lstdefinelanguage{psuedocode}{
%   keywords={typeof, new, true, false, catch, function, return, null, catch, switch, var, if, in, while, do, else, case, break},
%   keywordstyle=\color{blue}\bfseries,
%   ndkeywords={class, export, boolean, throw, implements, import, this},
%   ndkeywordstyle=\color{darkgray}\bfseries,
%   identifierstyle=\color{black},
%   sensitive=false,
%   comment=[l]{//},
%   morecomment=[s]{/*}{*/},
%   commentstyle=\color{purple}\ttfamily,
%   stringstyle=\color{red}\ttfamily,
%   morestring=[b]',
%   morestring=[b]"
% }

\usepackage{algpseudocodex}
%Book Stuff

%\usepackage{algorithm}
%\usepackage[noend]{algpseudocode}
%\makeatletter
%\def\BState{\State\hskip-\ALG@thistlm}
%\makeatother

% Figures & Drawings
\usepackage{graphicx, caption}
\usepackage{animate}
\usepackage{tikz}
\usepackage{float}
\usepackage{pict2e}
\usepackage{subcaption}

% Physics
\usepackage{physics}

% Mathematics
\usepackage{amsmath}
\usepackage{amssymb}
\usepackage{amsthm}
\usepackage{mathtools}
\usepackage{upgreek} % More Greek letters

% I don't really know what this is but I don't want to break shit
\usepackage{aliascnt}
\newaliascnt{eqfloat}{equation}
\newfloat{eqfloat}{h}{eqflts}
\floatname{eqfloat}{Equation}
\newcommand*{\ORGeqfloat}{}
\let\ORGeqfloat\eqfloat
\def\eqfloat{%
	\let\ORIGINALcaption\caption
	\def\caption{%
		\addtocounter{equation}{-1}%
		\ORIGINALcaption
	}%
	\ORGeqfloat
}
%}

% Bibliography (Citations) Formatting

%\usepackage{cite}
\usepackage{caption}
%\usepackage[backend=bibtex,style=verbose-trad2]{biblatex}
%works really really well, but no MLA format
%\usepackage[backend=biber]{biblatex}
%\bibliographystyle{apsrev4-1}
%\usepackage[backend=biber,style=mla]{biblatex} %Doesn't print all sources for some reason